\documentclass{article}

\usepackage{Sweave}
\begin{document}
\Sconcordance{concordance:deneme.tex:deneme.Rnw:%
1 2 1 1 0 2 1 1 2 1 0 6 1 1 2 2 1 1 3 1 0 1 5 3 0 1 3 1 0 12 1 1 2 1 1 %
1 2 1 7 6 0 1 1 1 7 6 0 1 2 2 1 1 2 4 0 1 3 3 1}


\begin{Schunk}
\begin{Sinput}
> library(ggplot2)
> library(reshape2)
> library(dplyr)
> library(data.table)
> library(stargazer)
> require(knitr)
> require(kableExtra)
> data = read.csv("Mexican_Data_Combined.csv") 
> subdata = as.data.frame(data)
> subdata = subdata[subdata$Actual.Total>=90000 & subdata$Actual.Total<=100000,]
> #subset the columns that we want to use to analyze Question 12
> subdata_12 = select(subdata, Respondent_ID, Institution_Type, P12_Distrito_1, P12_Distrito_2, P12_Distrito_3, P12_Distrito_4, P12_Distrito_5, P12_Distrito_6, P12_Distrito_7, P12_Distrito_8, P12_Distrito_9)
> #then to make the graph we need to change Districts 1-9 to there corresponding competitiveness level
> 
> #rearrange data, # I use the melt function because we want the column names to be on the x-axis. We want to exlude anything that is not a district. 
> subdata_12.melt = melt(subdata_12, id=c("Respondent_ID", "Institution_Type")) 
> #Rename the districts to level of competitiveness and Institution Type to English abbreviation.
> subdata_12.melt$variable = as.character(subdata_12.melt$variable)
> subdata_12.melt$Institution_Type = as.character(subdata_12.melt$Institution_Type)
> subdata_12.melt$variable[subdata_12.melt$variable == "P12_Distrito_1"] = "-.3"
> subdata_12.melt$variable[subdata_12.melt$variable == "P12_Distrito_2"] = "-.4"
> subdata_12.melt$variable[subdata_12.melt$variable == "P12_Distrito_3"] = "-.5"
> subdata_12.melt$variable[subdata_12.melt$variable == "P12_Distrito_4"] = "-.1"
> subdata_12.melt$variable[subdata_12.melt$variable == "P12_Distrito_5"] = "0"
> subdata_12.melt$variable[subdata_12.melt$variable == "P12_Distrito_6"] = "-.1"
> subdata_12.melt$variable[subdata_12.melt$variable == "P12_Distrito_7"] = "-.5"
> subdata_12.melt$variable[subdata_12.melt$variable == "P12_Distrito_8"] = "-.4"
> subdata_12.melt$variable[subdata_12.melt$variable == "P12_Distrito_9"] = "-.3"
> subdata_12.melt$Institution_Type[subdata_12.melt$Institution_Type == "RP"] = "PR"
> subdata_12.melt$Institution_Type[subdata_12.melt$Institution_Type == "MR"] = "SMD"
> subdata_12.melt = subdata_12.melt[complete.cases(subdata_12.melt), ] #to remove NAs
> subdata_12.melt = transform(subdata_12.melt, variable = as.numeric(variable)) #change variable from character to numerical
> Competitive = c()
> for (i in subdata_12.melt$variable) {
+   if (i == "0" | i == "-0.1") {
+     Competitive = c(Competitive, 1)
+   } else {
+     Competitive = c(Competitive, 0)
+   }
+ }
> Binary_IT = c()
> for (i in subdata_12.melt$Institution_Type) {
+   if (i == "SMD") {
+     Binary_IT = c(Binary_IT, 1)
+   } else {
+     Binary_IT = c(Binary_IT, 0)
+   }
+ }
> subdata_12.melt = cbind(subdata_12.melt, Binary_IT, Competitive)
> names(subdata_12.melt)[3]<-"Competitiveness"
> names(subdata_12.melt)[4]<-"District_Spending"
> ggplot(subdata_12.melt, aes(x = Competitiveness, y = District_Spending, color = Institution_Type)) + geom_jitter() + geom_smooth(method='lm', se = FALSE) + coord_cartesian(ylim = c(4000,20000)) + ggtitle("Mexican Students (ITAM)") + labs(y="District Spending", x = "Margin of Vote", color = "Electoral System") + scale_x_continuous(breaks = c(-.5, -.4, -.3, -.2, -.1, 0))
> 
\end{Sinput}
\end{Schunk}



\end{document}
